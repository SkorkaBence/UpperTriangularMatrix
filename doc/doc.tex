\documentclass[a4paper,12pt,twopage]{book}
\usepackage[utf8]{inputenc}
\usepackage{ragged2e}

\pagestyle{headings}

\begin{document}

Elnézést a dokumentum szörnyű formázásáért, ez az első \LaTeX dokumentumom.

\tableofcontents

\chapter{Feladat}

Valósítsa meg az egész számokat tartalmazó felsőháromszög mátrixtípust (a mátrixok a főátlójuk alatt csak nullát tartalmaznak)! Ilyenkor elegendő csak a főátló és afeletti elemeket reprezentálni egy sorozatban, amelyet egy dinamikus helyfoglalású tömbben helyezzünk el. Implementálja önálló metódusként a mátrix iedik sorának j-edik elemét visszaadó műveletet, valamint hatékony összeadás és szorzás műveleteket, továbbá a mátrix (négyzetes alakú) kiírását, és végül a másoló konstruktort és az értékadás operátort!

\chapter{Felsőháromszög mátrix osztály}

A feladat lényege egy felsőháromszög mátrix típusnak a megvalósítása.

\section{Típusérték halmaz}

\end{document}